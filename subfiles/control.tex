\chapter{Control Systems}
\label{ch:control-systems}
\section{Overview}
\newcommand{\itc}{I$^2$C}
The main goal of building the software systems around the pod, was creating a redundant yet lightweight and fast solution that would allow for safe communications with the vehicle and ensure complete control throughout pod launch.\\

The Software-system is built upon the Master-Slave design model to allow for maximum redundancy and safety at all times. The design is based on three components: Embedded-systems (Controls/Sensors Hub), Communication-systems (Master), and the Control-panel. 
Embedded-systems is responsible for all of the pod's internal interactions with the sensors and various health checks. 
Communication-systems is responsible for the complete pod control and has the ability to safely shut down the pod in case of an emergency. It is also responsible for the transportation of information from the vehicle to the control-panel to allow for live data visualization and remote control of the pod. 
Front-end is responsible for displaying all of the pod's vital readings and providing a control interface while the pod is in the tube. All code written for the software system of the pod is available for public access at \href{https://github.com/teamwaterloop}{Waterloop's GitHub Page} under MIT license.

\subsection{System Overview}
Two main design decisions of the Goose III's computational design are the parallelism of launch script execution and the Master-slave design. Waterloop has defined two clear computational layers - Master-Machines and Hubs. Master-machines are responsible for the complete control of the pod and the transfer of data from Hubs to the Control-Panel. While Control-panel provides with various control of the pod during the run, the entire launch script will be stored and executed by the Master Machines. By transferring launch script execution to on-board machines, network (WI-FI network provided by SpaceX for pod communications) inconsistencies are completely eliminated.

\begin{itemize}
    \item With \textbf{Master-Slave design}, there is a clear distribution of processing requirements for each compute machine, which allows for two layers of design. Having a clear distinction allows for a powerful abstraction of all the computing operations that must be performed on the pod, which makes the system highly scalable and easier to maintain.
    \item With \textbf{parallel execution} by the three Master-machines, in the case that an active machine comes offline for an unforeseen reason, the switch to the next machine is instantaneous. All three machines will be synced in their states described in the \reffig{fig:pod-state-diag}, where one Master-machine is active and the other two are in an idle mode, but maintain the same state as the Master-machine.
\end{itemize}

\begin{figure}
  \centering
  \includegraphics[width=\textwidth]{images/pod_state_diag.png}
  \caption{Pod-launch state diagram.}
  \label{fig:pod-state-diag}
\end{figure}

Based on \reffig{fig:software-diagram}, the entire system will incorporate 35 sensors, an ESC and a BMS, that will allow for a complete control over the pod and a live stream of data from all sensors.
% > 35 
% finalize # of sensors

\begin{figure}
  \centering
  \includegraphics[width=\textwidth]{images/pod_comp_arch.png}
  \caption{Goose 3 software architecture diagram}
  \label{fig:software-diagram}
\end{figure}

\section{Communication System}
Three major design decisions had to be made in order to develop a powerful, yet lightweight and fast communication pipeline to transfer packets from the low-level Hubs to a high-level Control-Panel.
\begin{itemize}
    \item Selecting the best packet type to create a fast and reliable communication channel; packet loss vs. packet speed trade-off
    \item Master-Slave design to allow for a powerful, yet modular and scalable system
    \item Developing an abstract package structure that allows for transfer of data from Slave to Master-machines with minimal overhead cost
\end{itemize}
We have decided to use a custom packet structure to transfer data at minimum overhead costs between Slave and Master-machines and a Golang server with QUIC packet streams to transmit data to the Control-Panel wirelessly

\subsection{Server and Network Protocols}
\label{subsec:comm-protocols}
\textbf{Golang vs. Node.js Server:} Go allows for language level concurrency with Go routines and has faster raw performance than our previous event-based Node.js server, decreasing overall latency. The server acts as a 'middle-man' communication channel between the low level Embedded-systems hubs and the high level Control-panel.\\

\noindent\textbf{TCP vs. QUIC Streams:} QUIC has built-in data reliability checks to eliminate packet loss and establishes connections much faster than TCP by removing the slow handshake process\footnote{\url{https://www.chromium.org/quic}}.
It is built on top of UDP and allows for multiplexing to provide extremely fast data transmission rates with minimal latency. After testing packet transmission speeds over a fairly stable wireless connection using TCP, UDP and QUIC streams, our results summarized in \reffig{fig:tcp_vs_quic} showed that multiplexed QUIC performed significantly faster (by up to approximately 20x depending on number of streams used) compared to both TCP and UDP.\\
\begin{figure}
\centering
\begin{subfigure}{0.7\textwidth}
    \includegraphics[width=\linewidth]{images/tcp-quic-bar-graph.png}
    \caption{Histogram}
\end{subfigure}
\begin{subfigure}{0.7\textwidth}
  \includegraphics[width=\linewidth]{images/streams-box-plots.png}
  \caption{Box Plots}
\end{subfigure}
	\caption{TCP vs. QUIC w/ no multiplexing vs. QUIC w/ multiplexing}
  \label{fig:tcp_vs_quic}
\end{figure}
As a result of a series of tests (\reffig{fig:tcp_vs_quic}), QUIC streams will be used to transmit control commands from the high level Control-panel (front-end client) and raw data back (from front-end client) to the on-board control elements. Multiple input and output streams can be opened across the server and client which allows for the simultaneous transmission of data packets from different sensors. The convention for all communications will be to use individual ordered output streams to transmit important sensor data packets that require ordered processing and individual input streams for all control packets coming in from the Control-panel.\\

\noindent\textbf{CANBUS:} Raw data packaged into Waterloop's custom defined packet structure will be transmitted through a CAN-bus network for communication between low-level Hubs and the communication server. As an alternative, \itc with all Hubs and Masters connected in parallel has been considered to be used instead of the CAN-BUS network. Two main factors have been used to select CAN-BUS over an \itc network:
\begin{itemize}
	\item Distance limitation of the \itc network. Usually used for internal PCB components, not for vehicle long communication.
    \item Speed limitation of the \itc network. CAN-BUS can allow up to \SI{1}{Mbps} packet transfer speed on much longer distances\footnote{\url{http://www.can-wiki.info/doku.php}}.
\end{itemize}

\paragraph{CAN-BUS Protocol Description}
CAN is a protocol that was designed for vehicle controller commutations to transmit information in a network without a host computer. It is a message based protocol with built-in priority and error checking. The priority allows for nodes to send higher priority messages, which places them higher in the message priority queue. Error checking allows for frames to check for losses and discard if they contain errors.

\subsection{Communication-System redundancy measures}
\qquad\textbf{Parallel Master-Machines} In order to ensure instantaneous recovery options, Waterloop will run three identical Master-Machines, with one being active and the others inactive. The active machine will be responsible for execution of bidirectional communication and execution of all control commands, as well as the pod-launch script. In the mean time, the inactive machine will stay up to date with the state of the active machine. In case of the failure of the active machine, measures such as heartbeat will allow for instant detection of device failure and and a switch to one of the idle machines into an active state. Control-Panel, using heart-beat as well, will detect the change of active machine and will adjust by restarting the failed Master-machine and putting it into inactive state once the reboot is complete.\\

\textbf{Distributed System} Communication-system's distributed system will be achieved by running three Raspberry Pi controllers at the same time using a Docker Swarm. This will use the Raft consensus algorithm with an internal distributed state store to determine validity of data for each node and verify consistent data across all three controllers. The nodes will communicate through \itc.\\
    The Docker Swarm allows for two types of nodes, the manager node and worker node. Manager nodes distribute tasks among other worker nodes, and send data to the client through the communication protocols described in \refsubsec{subsec:comm-protocols}. Worker nodes perform the same tasks as other workers to reach consensus and can have tasks distributed among them.\\
    Initially at runtime, the pod will have a single manager node and two worker nodes running concurrently, where in the case of a manager failure, one of the worker nodes will be promoted to a manager.\\
    Controller failure will be monitored through a heartbeat managed by Docker which uses a small timeout for nodes to respond.
    
\section{Control Packet Format}
    \subsection{JSON structure}
    \label{subsec:json}
    All communications between on-board Master-machines and Control-panel are JSON packets serialized from Go structures in the following format:
	\begin{minted}{go}
type CommPacketJson struct {
    Time int64      `json:"time"`
    Type string     `json:"type"`
    Name string     `json:"name"`
    Data []float32  `json:"data"`
}
    \end{minted}
    \begin{table}[H]
        \centering
        \begin{tabular}{@{}lp{4in}l@{}} \toprule
            Property & Description & Example\\ \midrule
            \texttt{time}
            & Time of packet creation, in milliseconds since Unix epoch & \tabxmintinline{json}{1513452619442}\\
            \texttt{type} & Packet type, see possible packet types & \tabxmintinline{json}{"sensor"}\\
            \texttt{name} & Specific name of packet, explicitly describing role and origin of packet (i.e. data from a specific sensor) & \tabxmintinline{json}{"accel1"}\\
            \texttt{data} & 3-tuple of float32 values representing packet data & \tabxmintinline{json}{[32.2323, 12.22, 23.11]}\\ \bottomrule
        \end{tabular}
        \caption{Description binary packet fields}
    \end{table}
    Example of a serialized JSON:
    \begin{minted}{json}
{
    "time": 1513452619442,
    "type": "sensor",
    "name": "accel",
    "data": [0.00, 0.00, 0.00]
}
    \end{minted}
    \subsection{Binary Packet Structure}
    Communication between slaves (Arduino) and on-board masters (Raspberry Pi) consists of 64-bit CAN data packets, the structure of which is described below. Packets are checked with a CRC-8 checksum $(0$x$97 = x^8 + x^5 + x^3 + x^2 + 1)$.
    \begin{table}[H]
        \centering
        \begin{tabulary}{\textwidth}{@{}LLlll@{}} \toprule
            {[0:2]} & {[3:9]} & {[10:27]} & {[28:45]} & {[46:63]}\\ \midrule
            3 bits & 7 bits & 18 bits & 18 bits & 18 bits\\
            Packet Type & Packet Name & Data value 1 & Data value 2 & Data value 3\\ 
            \\
            see \ref{par:packet-name-repr} & see \ref{par:packet-name-repr} & & see \ref{par:data-repr} \\ \bottomrule
        \end{tabulary}
        \caption{Binary packet structure breakdown}
    \end{table}
    \paragraph{Packet Type Representation}
    \label{par:packet-type-repr}
    Each packet type is represented by 3 bit encoding
    \begin{table}[H]
        \centering
        \begin{tabular}{@{}ll@{}} \toprule
            Bits & Type\\ \midrule
            000 & sensor\\
            001 & command\\
            010 & state\\
            011 & log\\ \bottomrule
        \end{tabular}
        \caption{Packet types mapped to binary representation}
    \end{table}

    \paragraph{Packet Name Representation}
    \label{par:packet-name-repr}
    Each packet name is represented by 7 bits. All of the 35 sensors on-board of the pod will be represented using the 7 bit binary codes. Having $2^7=127$ possible sensor encodings, there are more than enough combinations to cover all on-board sensors of the Goose 3 pod.

    \paragraph{Data representation}
    \label{par:data-repr}
    Data values are floats represented by 18 bits as follows:
    \begin{table}[H]
        \centering
        \begin{tabular}{@{}lll@{}} \toprule
            Sign & Exponent & Significand\\ \midrule
            1 bit & 5 bits & 12 bits\\ \bottomrule
        \end{tabular}
        \caption{Structure of 18-bit float. Significand has 13 bits of precision with 12 explicitly stored.}
    \end{table}
    This representation follows from the IEEE specifications for half and full precision floating point numbers. This allows us a max value of \texttt{65504} and a minimum positive normal of about \texttt{1.5258789e-5} with approximately 4 significant digits.
    \section{Communication Packet Types and Usage}
    Packets can be exchanged between Embedded-systems and Controls-panel depending on type.
    \subsection{Command}
    Command packets are unidirectional from Control-panel to the control elements on the pod.
\def\tabxmintinline#1#2{%
\ifx\@footnotetext\TX@trial@ftn
\detokenize{#2}%
\else
\mintinline{#1}{#2}%
\fi}
\makeatother
    \setmintedinline[json]{breaklines}
	\begin{table}[H]
        \centering
        \begin{tabular}{@{}lllp{3.8in}@{}} \toprule
            Name & Code & Value(s) & Example\\ \midrule
            Brake & \texttt{brk} & None & \tabxmintinline{json}{{"time": ..., "type": "command", "name": "brk", "data": []}}\\
            Emergency & \texttt{emg} & None & \tabxmintinline{json}{{"time": ..., "type": "command", "name": "emg", "data": []}}\\
            Speed & \texttt{spd} & \texttt{0.00 - 100.00} & \tabxmintinline{json}{{"time": ..., "type": "command", "name": "spd", "data": [100.00]}}\\
            Start Pod & \texttt{start} & None & \tabxmintinline{json}{{"time": ..., "type": "command", "name": "start", "data": []}}\\
            Stop Pod & \texttt{stop} & None & \tabxmintinline{json}{{"time": ..., "type": "command", "name": "stop", "data": []}}\\
            Health Check & \texttt{health} & None & \tabxmintinline{json}{{"time": ..., "type": "command", "name": "health", "data": []}}\\
            Coast & \texttt{coast} & None & \tabxmintinline{json}{{"time": ..., "type": "command", "name": "coast", "data": []}}\\
            Accelerate & \texttt{accel} & None & \tabxmintinline{json}{{"time": ..., "type": "command", "name": "accel", "data": []}}\\ \bottomrule
        \end{tabular}
        \caption{Control-panel to Pod command packet examples}
    \end{table}

\subsection{Sensor}
    Sensor packets are unidirectional and are sent from pod's Embedded-system to the Control-panel where all data is visualized in series of graphs.
    \begin{table}[H]
        \centering
        \begin{tabular}{@{}lllp{3.65in}@{}} \toprule
            Name & Code & Value(s) & Example \\ \midrule
            {[Sensor Name]} & \texttt{[sensor code]} & None & \tabxmintinline{json}{{"time": ..., "type": "sensor", "name": "accel", "data": [12.11, 12.34, 43.21]}} \\ \bottomrule
        \end{tabular}
        \caption{}
    \end{table}

\subsection{Log}
    Hubs and Master-machines are capable of sending Log messages that are displayed on the Control-panel and stored in Pod's non-volatile memory. The stored log messages can be accessed post-launch for launch analysis.
    \begin{table}[H]
        \centering
        \begin{tabular}{@{}lllp{3.9in}@{}} \toprule
            Name & Code & Value(s) & Example \\ \midrule
            {[Any name]} & \texttt{[any code]} & Any values & \tabxmintinline{json}{{"time": ..., "type": "log", "name": "accel", "data": [12.11, 12.34, 43.21]}} \\ \bottomrule
        \end{tabular}
        \caption{Example of Log message packet}
    \end{table}
    
\subsection{State}
    Sent from Control-panel or Embedded-systems to update state of Master-Machines
    \begin{table}[H]
        \centering
        \begin{tabular}{@{}lllp{4.25in}@{}} \toprule
            Name & Code & Value(s) & Example \\ \midrule
            Braking & \texttt{brk} & \texttt{0, 1} & \tabxmintinline{json}{{"time": ..., "type": "state", "name": "brk", "data": [1]}} \\ \bottomrule
        \end{tabular}
        \caption{Example of state packet}
    \end{table}
 
    \subsection{Data Logging and Blackbox}
    In the short-term, Generic data logging will be implemented by storing all transmitted sensor data and pod commands into JSON files onto the Raspberry Pi as a blackbox. Data from each sensor will be written to separate files to allow for future data processing and analysis of patterns.\\
    As a long-term data storage solution, Waterloop will use AWS S3 to store large amounts of data without sacrificing accessibility. For post-launch data analytics, a local Grafana UI will be built to provide quick and modular data visualization tools, thus allowing the team to debug every single pod launch.
       
        
\section{Navigation System}
Navigation system will be running on all Master-machines and will use a variety of inputs to generate a precise location of the pod inside the tube. The system will be based on two IMUs, a Shaft Encoder and a Timer. Since raw data from accelerometers has a lot of variability due to the sensitivity of sensors, the IMU outputs will be passed through a Kalman filter. \reffig{fig:navigation-system} Describes the flow of data within the Navigation-system.

\begin{figure}
    \centering
    \includegraphics[width=\textwidth]{images/navigation_system.png}
    \caption{Simulation of data flow within the Navigation-system on the pod.}
    \label{fig:navigation-system}
\end{figure}


