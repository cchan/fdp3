\documentclass[main.tex]{subfile}
\begin{document}
    \chapter{Introduction}
    \section{Our Team}
    We are Waterloop, a student design team … demographics…\\

    We are the only all-Canadian organization to have built a functioning Hyperloop prototype, and we intend to continue to represent Canadian innovation through participation in the 2018 Hyperloop Competition. [something something trudeau and gg]\\

    Our mission has several components:
    To strongly represent Waterloo as a hub for Canadian innovation through the SpaceX Hyperloop Competition.
    \begin{itemize}
        \item To teach technical skills, strong engineering practices, and teamwork.
        \item To reach out to inspire and cultivate young innovators from all backgrounds.
        \item To make possible a future that is worth building.
    \end{itemize}
    Who we are\\
    Mission\\
    Marketing, outreach, stats on how impactful we are - blogs and team member profiles, heforshe\\
    Brief mention of financials, convince them we can fund it

    \section{Team Member Directory}
    \begin{center}
        \textsf{Board of Advisors}
    \end{center}
    \begin{multicols}{2}
        \begin{center}
            Serhiy Yarusevych\\
            a\\
        \end{center}
        \columnbreak
        \begin{center}
            Victor Qian\\
            b
        \end{center}
    \end{multicols}
    \begin{center}
        \textsf{Team Leadership}
    \end{center}
    \begin{multicols}{4}
        \begin{center}
            Benjamin Tonita\\
            \begin{small}
                -Integration Engineer-\\
            \end{small}
        \end{center}
        \columnbreak    \begin{center}
            Clive Chan\\
            \begin{small}
                -Technical Director-\\
            \end{small}
        \end{center}
        \columnbreak    \begin{center}
            Jason Pan\\
            \begin{small}
                -Administrative Director-\\
            \end{small}
        \end{center}
        \columnbreak
        \begin{center}
            Jimmy Zhou\\
            \begin{small}
                -Integration Engineer-
            \end{small}
        \end{center}
    \end{multicols}
    \begin{center}
        \textsf{Team Members}\\
        \hfill\\
        \begin{tabular}{| c | c | c | c |}
            \hline
            &   &   & \\
            \hline
            &   &   & \\
            \hline
        \end{tabular}
    \end{center}
    \begin{itemize}
        \item Advisors: Serhiy Yarusevych, Victor Qian
        \item Directors: Jason Pan, Clive Chan
        \item Integration Engineers: Jimmy Zhou, Benjamin Tonita
        \item Administrative Leads: Nicholas Jelich, Natalia Zigante, Nafee Hasan, Aditya Arora
        \item Technical Leads: William Ngana, Deep Dhillon, Ruslan Nikolaevra, Urooj Khaleeli, Jimmy Zhou, Benjamin Tonita, Chawthri Kanagarasa
    \end{itemize}

    \section{Sponsors}
    Thanks to all our sponsors, past and present

    \begin{itemize}
        \item Which logos? Sponsorship team should decide
        \item We’ll have more soon
    \end{itemize}

    \section{Acknowledgements}

    \begin{itemize}
        \item Advisors: Serhiy, Victor
        \item Waterloo Formula Electric for guidance on embedded and electrical
        \item Sandra \& Sedra Student Design Centre
        \item Various other things
    \end{itemize}

    \chapter{Top-Level Design}
    Labeled CAD (Exploded? Semitransparent? Color coded?)\\

    Our pod design process for Competition III began with the goal of building the fastest possible pod. Based purely on specific power and energy of lithium ion batteries (and perhaps supercapacitors), it is theoretically possible to achieve speeds above Mach 1 within the 1-mile Hyperloop test track. Our pod this year is designed to achieve 100 m/s, and we are working gradually toward higher and higher speeds for future competitions.


    \section{Summary of design}
    Our pod consists of the following major components:
    \begin{itemize}
        \item Propulsion: Friction drive (wheels)
        \item Braking: High speed eddy current braking, low speed caliper brakes
        \item Lateral stability: Freely spinning wheels
        \item Frame: Aluminum ladder frame
        \item
    \end{itemize}
    Uniqueness/Novelty\\
    Changes from PDB and reasoning for changes\\
    If reusing a system in whole or part, explain changes\\

    \subsection{Size and Mass}
    The size of the pod is primarily dictated by the size of the propulsion system and the size of the I-beam.\\
    Table: pod dimensions, mass by subsystem\\
    Overall weight distribution/Center of mass\\
    The cross-section of the pod is small enough that aerodynamic effects like the Kantrowitz limit\\
    are negligible.\\

    \subsection{Energy Storage and Usage}
    Where is energy stored onboard the pod?
    \begin{itemize}
        \item Huge batteries
        \item Air tanks
        \item Lateral/EC Brakes springs?
    \end{itemize}
    Where is energy used onboard the pod?
    \begin{itemize}
        \item Friction drive
        \item Electronic systems (electrically isolated)
    \end{itemize}

    \subsection{Overall pod thermal profile}
    How are various parts of the pod cooled? (summarize)

    \subsection{Predicted Pod trajectory}
    Speed versus distance graph

    \section{Scalability}
    Our pod design is essentially the simplest possible design for a Hyperloop. However, it is difficult to scale to high-subsonic speeds.
    “gain experience with high power systems and high speed stability and braking systems.”\\

    Preliminary analysis on scalability to an operational Hyperloop with respect to:
    \begin{itemize}
        \item System size (increased tube length, tube diameter, and Pod size)
        \item Cost (both production and maintenance)
        \item Estimated Pod mass and cost if built full-scale
        \item Maintenance (e.g. not requiring specialized alignment tools, hot-swappable subsystems)
        \item Power, cooling, safety, etc.
        \item Propulsion effectiveness
    \end{itemize}
\end{document}