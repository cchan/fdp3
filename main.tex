\documentclass[hidelinks, twoside]{report}
\usepackage[utf8]{inputenc}
\usepackage[toc,page]{appendix}
\usepackage[letterpaper, margin=1in]{geometry}
\usepackage{tocbibind}
\usepackage{datetime}
\usepackage{multicol}
\usepackage[draft]{graphicx}
\usepackage{multirow}
\usepackage{subcaption}
\usepackage{float}
\usepackage{siunitx}
\usepackage{amsmath}
\usepackage{booktabs}
\usepackage{makecell}
\usepackage{subfiles}
\usepackage{textcomp}
\usepackage{hyperref}
\usepackage{tabularx}
\usepackage{tabulary}
\usepackage{color}
\usepackage[table]{xcolor}
\usepackage{fancyhdr}
\usepackage{etoc}
\usepackage{minted}
\usepackage{etoolbox}
\usepackage{lscape}
\usepackage{enumitem}
\usepackage{titlesec}
\usepackage{incgraph,tikz}

\titleformat{\chapter}[display]
{\normalfont\huge\bfseries}{\chaptertitlename\ \thechapter}{20pt}{}
\titlespacing*{\chapter}{0pt}{-20pt}{0pt}

\pagestyle{fancy}
\renewcommand{\etocaftertitlehook}{\thispagestyle{fancy}}
\renewcommand{\etocaftertochook}{\thispagestyle{fancy}}
\patchcmd{\chapter}{\thispagestyle{plain}}{\thispagestyle{fancy}}{}{}
\fancyhf{}
\fancyhead[LE,RO]{Competition 3 Final Design Package}
\fancyhead[RE,LO]{Waterloop}
\fancyfoot[LE,RO]{\thepage}
\fancyfoot[CE,CO]{
	\begin{minipage}{.18\textwidth}
        \centering
        \includegraphics[width=\linewidth,height=15pt,keepaspectratio]{images/w_logo_transparent.png}
    \end{minipage}
}

\author{Team Waterloop}

\newcommand{\refeq}[1]{\hyperref[#1]{Equation \ref*{#1} on page \pageref*{#1}}}
\newcommand{\refsec}[1]{\hyperref[#1]{Section \ref*{#1} on page \pageref*{#1}}}
\newcommand{\refsubsec}[1]{\hyperref[#1]{Subsection \ref*{#1} on page \pageref*{#1}}}
\newcommand{\refsubsubsec}[1]{\hyperref[#1]{Subsubsection \ref*{#1} on page \pageref*{#1}}}
\newcommand{\refpara}[1]{\hyperref[#1]{Paragraph \ref*{#1} on page \pageref*{#1}}}
\newcommand{\reffig}[1]{\hyperref[#1]{Figure \ref*{#1} on page \pageref*{#1}}}
\newcommand{\reftab}[1]{\hyperref[#1]{Table \ref*{#1} on page \pageref*{#1}}}
\newcommand{\refapp}[1]{\hyperref[#1]{Appendix \ref*{#1}} on page \pageref*{#1}}
% \setlength{\parindent}{0pt}
\setcounter{secnumdepth}{5}
\setcounter{tocdepth}{5}
\title{Waterloop Final Design Package}

\newdate{date}{12}{01}{2018}
\date{\displaydate{date}}

\renewcommand{\contentsname}{Table of Contents}
\let\oldparagraph\paragraph

\renewcommand{\paragraph}[1]{\oldparagraph{#1}\mbox{}\\}

% % Hack to make minted work inside tabularx
\makeatletter
\def\tabxmintinline#1#2{%
\ifx\@footnotetext\TX@trial@ftn
\detokenize{#2}%
\else
\mintinline{#1}{#2}%
\fi}
\makeatother

\begin{document}
    %%%---%%%---%%%---%%%---%%%---%%%---%%%---%%%---%%%---%%%---%%%---%%%---%%%
    %   TITLEPAGE
    %
    %   due to variety of titlepage schemes it is probably better to make titlepage manually
    %
    %%%---%%%---%%%---%%%---%%%---%%%---%%%---%%%---%%%---%%%---%%%---%%%---%%%
    \thispagestyle{empty}
    
    \incgraph[documentpaper]
  [width=\paperwidth,height=\paperheight]{images/front_off-main_front.png}

    Revised \displaydate{date}\\

    Prepared by the Waterloop Hyperloop Team.

    Prepared for Space Exploration Technologies Corp. for the Hyperloop Pod Competition.\\

    \copyright \ 2018 by Waterloop.

    All rights reserved. Not for public distribution. Yet.\\

    Document approved for submission by the Waterloop executive team on January 12, 2018.

    \tableofcontents
    \newpage

	% WE ARE NOT INCLUDING THE QUICK REFERENCE.
 	% \subfile{subfiles/00_quick_reference.tex}
    \incgraph[documentpaper]
  	[width=\paperwidth,height=\paperheight]{images/intro.png}
    \subfile{subfiles/01_introduction.tex}
    \incgraph[documentpaper]
  	[width=\paperwidth,height=\paperheight]{images/front_off_propulsion.png}
    \subfile{subfiles/02_propulsion.tex}
    \incgraph[documentpaper]
  	[width=\paperwidth,height=\paperheight]{images/front_off_eddy.png}
    \subfile{subfiles/03_eddy_current_brakes.tex}
    \subfile{subfiles/04_lateral.tex}
    \incgraph[documentpaper]
  	[width=\paperwidth,height=\paperheight]{images/front_off_frame.png}
    \subfile{subfiles/05_frame.tex}
    \subfile{subfiles/06_shell.tex}
    \incgraph[documentpaper]
  	[width=\paperwidth,height=\paperheight]{images/front_off_electrical.png}
    \subfile{subfiles/07_electrical.tex}
    \incgraph[documentpaper]
  	[width=\paperwidth,height=\paperheight]{images/front_off_embedded.png}
    \subfile{subfiles/08_embedded.tex}
    \incgraph[documentpaper]
  	[width=\paperwidth,height=\paperheight]{images/front_off_control_system.png}
    \subfile{subfiles/09_control.tex}
    \incgraph[documentpaper]
  	[width=\paperwidth,height=\paperheight]{images/front_off_top-level.png}
    \subfile{subfiles/10_next_steps.tex}

    \begin{appendices}
        \addtocontents{toc}{\protect\setcounter{tocdepth}{1}}
        \makeatletter
        \addtocontents{toc}{
            \begingroup
            \let\protect\l@chapter\protect\l@section
            \let\protect\l@section\protect\l@subsection
        }
        \makeatother
        
        \chapter{Hazards \& FMEA}

Hazardous materials can be released due to any of these situations: rupturing lithium cells, short-circuiting lithium cells that could cause a fire, fires that may cause lithium cells to explode, or piercing/ breaking the protective layers of any cell. If a lithium battery explodes, there can be released molten lithium shrapnel.\\

\noindent \underline{List of hazards and associated hazardous materials:}\\

\noindent TOXIC (Harmful if inhaled or swallowed):\\
\noindent Thionyl chloride, Bromine gas, Chlorine dioxide gas, Sulfur dioxide gas, Sulfuryl chloride gas.\\

\noindent CORROSIVE (Causes skin, digestive and respiratory tract burns and eye damage):\\
\noindent Thionyl chloride, Bromine gas, Chlorine dioxide gas, Hydrochloric acid, Sulfur dioxide gas, Sulfuryl chloride gas, Lithium hydroxide.\\

\noindent FLAMMABLE:\\
\noindent Hydrogen gas\\

\noindent OXIDIZING MATERIAL (May cause or 
intensify fire; oxidizer):\\
\noindent Chlorine dioxide gas\\

\noindent \underline{General procedures in keeping others safe:}\\
    
\noindent Make sure anyone wanting to inspect the Waterloop pod, or any of its components are accompanied by a trained Waterloop member. In case of any emergency, warn others and immediately leave to a safer area. Report the emergency and attend to any person exposed to any leaked hazardous materials. Seek medical attention, if required.\\

\noindent \underline{Failure Mode and Emergency Analysis}\\

\begin{table}
\centering
  \begin{tabular}{@{}p{2cm}p{2cm}p{3cm}p{1cm}p{3cm}p{3cm}@{}} \toprule
    Function & Failure Type & Potential Impact & Severity & Potential Causes & Detection Mode \\ \midrule
    Air bag suspension & Leak / burst & Poor stability & 9 & Defective airbag & None\\
    Belt & Snap & No power delivery & 9 & Over-tension, defective belt & RPM spike and speed decreases\\
    Motor cooling & Leak & Motor overheating & 6 & Defective hose & Temperature sensor\\
    Pneumatic actuator & Leak/Burst & Insufficient pressure/force & 9 & Over-pressurization, defective & Pressure sensor \\
    Wheels & Polyurethane wearing off & More resistance during propulsion, lower speed & 7 & Polyurethane not able to withstand high speed & None \\
    Bearings & Degradation & Increase in friction / resistance, overheating & 7 & Improper lubrication, defective bearings & None \\
    Shaft & Deformation & Less effective transmission system & 8 & Excessive force on transmission system & None \\ \bottomrule
  \end{tabular}
  \caption{FMEA Propulsion}
  \label{table:fmea-propulsion}
\end{table}

\begin{table}[]
\centering
\caption{Electrical Failsafe Circuit}
\label{my-label}
\begin{tabular}{ccc|ccccc}
\hline
\multicolumn{3}{c}{Input}     & \multicolumn{5}{c}{Output} \\ \hline
Main Battery & 24V 1 & 24V 2   & All Arduino Hubs & EC Brakes Engage & EC Brakes Disengage & Friction Brakes & Friction Drive \\ \hline
1            & 1     & 1       & 1                & 1 & 1         & 1               & 1              \\
1            & 1     & 0       & 1                & 0 & 0        & 1               & 0              \\
1            & 0     & 1       & 1                & 0 & 0        & 1               & 0              \\
1            & 0     & 0       & 0                & 0 & 0        & 0               & 0              \\
0            & 1     & 1       & 1                & 0 & 0        & 1               & 0              \\
0            & 1     & 0       & 1                & 0 & 0        & 1               & 0              \\
0            & 0     & 1       & 1                & 0 & 0        & 1               & 0              \\
0            & 0     & 0       & 0                & 0 & 0        & 0               & 0              \\ \hline
\end{tabular}
\end{table}

\begin{table}
\centering
  \begin{tabular}{@{}p{2cm}p{2cm}p{3cm}p{1cm}p{3cm}p{3cm}@{}} \toprule
    Function & Failure Type & Potential Impact & Severity & Potential Causes & Detection/Mitigation \\ \midrule
    \multirow{2}{*}{Battery} & Overheat & Fire & 10 & Short Circuit & Temperature sensors and Fuses \\
    & Disconnection & Power Loss & 9 & Poor Connection/Wire Severed & 24V Battery Redundancy and Safety Circuit \\
    BMS & Disconnection & Loss of cell voltage data from pod status telemetry frame & 5 & Poor Connection/Wire Severed & Telemetry data \\
    \multirow{2}{*}{IMU} & Disconnection & Loss of acceleration/velocity data & 7 & Poor Connection/Wire Severed & Sensor Redundancy \\
    & Erroneous acceleration/velocity data & Inaccurate Pod State Estimation & 9 & Faulty hardware & Sensor Redundancy and Data Fusion \\


     \bottomrule
  \end{tabular}
  \caption{FMEA Electrical}
  \label{table:fmea-electrical}
\end{table}

\chapter{Bill of Materials}
        \label{app:bom}
        \subfile{subfiles/bom.tex}
        \addtocontents{toc}{\endgroup}
\end{appendices}


\end{document}
